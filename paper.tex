% Template for ICIP-2013 paper; to be used with:
%          spconf.sty  - ICASSP/ICIP LaTeX style file, and
%          IEEEbib.bst - IEEE bibliography style file.
% --------------------------------------------------------------------------
\documentclass{article}
\usepackage{spconf,amsmath,graphicx}

% Example definitions.
% --------------------
\def\x{{\mathbf x}}
\def\L{{\cal L}}

% Title.
% ------
\title{Investigation on Parallelization of Deep Neural Network Training using Multiple GPUs}
%
% Single address.
% ---------------
\name{Hang Su$^1$$^2$, Haoyu Chen$^1$, Nelson Morgan$^2$}
\address{$^1$ International Computer Science Institute, Berkeley, California, US \\
$^2$ Dept. of Electrical Engineering \& Computer Science, University of California, Berkeley, CA, USA \\
{\small \tt \{suhang3240@gmail.com, williamchenthu@gmail.com\}}
}
%
% For example:
% ------------
%\address{School\\
%	Department\\
%	Address}
%
% Two addresses (uncomment and modify for two-address case).
% ----------------------------------------------------------
%\twoauthors
%  {A. Author-one, B. Author-two\sthanks{Thanks to XYZ agency for funding.}}
%	{School A-B\\
%	Department A-B\\
%	Address A-B}
%  {C. Author-three, D. Author-four\sthanks{The fourth author performed the work
%	while at ...}}
%	{School C-D\\
%	Department C-D\\
%	Address C-D}
%
\begin{document}
%\ninept
%
\maketitle
%
\begin{abstract}
In this paper we introduce Butterfly mixing to parallel training of deep neural networks (DNN). Parallelization is done
in a model averaging manner. Data is partitioned and distributed to different nodes for local model update, and
model averaging (reduce) is done every few minibatches. We compare several different reduce stratgies, including
all reduce, butterfly mixing, ring reduce and hopping ring reduce. We show that all these methods can effectively
speed up neural network training. On swithboard data, a xx times speed up is achieved using 16 gpus.

\end{abstract}
%
\begin{keywords}
Parallel training, multiple GPUs, neural network, butterfly mixing
\end{keywords}
%
\section{Introduction}
\label{sec:intro}
Deep Neural Networks (DNN) has shown its effeciveness in several machine learning tasks, espencially in speech
recognition. The large model size and massive training examples make DNN a powerful model for classification. However,
these two factors also slow down the training procedure of DNNs.

Parallelization of DNN training has been a popular topic since the revive of neural networks. Several different strategies
have been proposed to tackle this problem. Multiple thread CPU parallelization and single CUDA GPU implementation are compared
in \cite{scanzio2010parallel,vesely2010parallel}, and they show that single GPU could beat 8 cores CPU by a factor of 2.

Optimality for parallelization of DNN training was analyzed in \cite{seide2014parallelizability}, and based on the analysis, 
a gradient quantization approach was proposed to minimize communication cost \cite{seide20141}.

DistBelief proposed in \cite{dean2012large} reports a speed up of 2.2x using 8 CPU coresthan using a
single machine.


Asynchronous SGD using multiple GPUs achieved a 3.2x speed-up on 4 GPUs \cite{zhang2013asynchronous}.

A pipeline training approach was propoased in \cite{chen2012pipelined} and a 3.3x speedup was achieved using 4 GPUs, but this
method does not scale beyond number of layers in the neural network.

A speedup of 6x to 14x was achieved using 16 GPUs on training convolutional neural networks \cite{coates2013deep}. In this approach,
each GPU is responsible for a partition of the neural network. This approach is more useful for image classification where 
local structure of the neural network could be exploited.

Distributed model averaging using CPUs is proposed in \cite{zhang2014improving},
and a further improvement is done using natural gradient \cite{povey2014parallel}.


Butterfly mixing was proposed in \cite{zhao2013butterfly} to interleave communication with computation.


\section{Data parallelization and Model Averging}

\section{Different Reduce Strategies}

\subsection{Butterfly Mixing}

All printed material, including text, illustrations, and charts, must be kept
within a print area of 7 inches (178 mm) wide by 9 inches (229 mm) high. Do
not write or print anything outside the print area. The top margin must be 1
inch (25 mm), except for the title page, and the left margin must be 0.75 inch
(19 mm).  All {\it text} must be in a two-column format. Columns are to be 3.39
inches (86 mm) wide, with a 0.24 inch (6 mm) space between them. Text must be
fully justified.

\subsection{Ring \& Hopping Ring Reduce}

The paper title (on the first page) should begin 1.38 inches (35 mm) from the
top edge of the page, centered, completely capitalized, and in Times 14-point,
boldface type.  The authors' name(s) and affiliation(s) appear below the title
in capital and lower case letters.  Papers with multiple authors and
affiliations may require two or more lines for this information. Please note
that papers should not be submitted blind; include the authors' names on the
PDF.

\section{Natural Gradient for Model Update}

\section{Experimental Results}
\subsection{Reduce Frequency}

\subsection{Reduce type}

\subsection{Scaling factor}

\section{Conclusion}


\section{Acknowledgements}
We would like to thank Forrest Iandola for helpful suggestion.

% References should be produced using the bibtex program from suitable
% BiBTeX files (here: strings, refs, manuals). The IEEEbib.bst bibliography
% style file from IEEE produces unsorted bibliography list.
% -------------------------------------------------------------------------
\bibliographystyle{IEEEbib}
\bibliography{paper}

\end{document}
