% Template for ICIP-2013 paper; to be used with:
%          spconf.sty  - ICASSP/ICIP LaTeX style file, and
%          IEEEbib.bst - IEEE bibliography style file.
% --------------------------------------------------------------------------
\documentclass{article}
\usepackage{spconf,amsmath,graphicx}

% Example definitions.
% --------------------
\def\x{{\mathbf x}}
\def\L{{\cal L}}

% Title.
% ------
\title{Parallel Training of Deep Neural Network using Model Averaging and Butterfly Mixing}
%
% Single address.
% ---------------
\name{Hang Su$^{1,2}$, Haoyu Chen$^1$, Haihua Xu$^3$}
\address{$^1$ International Computer Science Institute, Berkeley, California, US \\
$^2$ Dept. of Electrical Engineering \& Computer Science, University of California, Berkeley, CA, USA \\
$^3$ Nanyang Technological University, Singapore \\
{\small \tt \{suhang3240@gmail.com\}}
}
%
% For example:
% ------------
%\address{School\\
%	Department\\
%	Address}
%
% Two addresses (uncomment and modify for two-address case).
% ----------------------------------------------------------
%\twoauthors
%  {A. Author-one, B. Author-two\sthanks{Thanks to XYZ agency for funding.}}
%	{School A-B\\
%	Department A-B\\
%	Address A-B}
%  {C. Author-three, D. Author-four\sthanks{The fourth author performed the work
%	while at ...}}
%	{School C-D\\
%	Department C-D\\
%	Address C-D}
%
\begin{document}
%\ninept
%
\maketitle
%
\begin{abstract}
In this paper we apply model averaging and butterfly mixing to parallel training of deep neural network (DNN). 
Parallelization is done in a model averaging manner. Data is partitioned and distributed to different nodes for 
local model updates, and model averaging (reduce) is done every few minibatches. 

We use multiple GPUs for data parallelization, and Message Passing Interface (MPI) for communication between nodes. 
Natrual Gradient for Stochasitc Gradient Descent (NG-SGD) approach is used to ensure better convergence in this framework. On the
300h Swithboard dataset, a 12 times speed up is achieved using 16 gpus with no decoding accuracy loss. We also 
compare all-reduce strategy with butterfly mixing, and show that butterfly mixing can further save communication costs.

\end{abstract}
%
\begin{keywords}
Parallel training, model averaging, deep neural network, butterfly mixing
\end{keywords}
%
\section{Introduction}
\label{sec:intro}
Deep Neural Networks (DNN) has shown its effeciveness in several machine learning tasks, espencially in speech
recognition. The large model size and massive training examples make DNN a powerful model for classification. However,
these two factors also slow down the training procedure of DNNs.

Parallelization of DNN training has been a popular topic since the revive of neural networks. Several different strategies
have been proposed to tackle this problem. Multiple thread CPU parallelization and single CUDA GPU implementation are compared
in \cite{scanzio2010parallel,vesely2010parallel}, and they show that single GPU could beat 8 cores CPU by a factor of 2.

Optimality for parallelization of DNN training was analyzed in \cite{seide2014parallelizability}, and based on the analysis, 
a gradient quantization approach was proposed to minimize communication cost \cite{seide20141}.

DistBelief proposed in \cite{dean2012large} reports a speed up of 2.2x using 8 CPU coresthan using a
single machine.

Asynchronous SGD using multiple GPUs achieved a 3.2x speed-up on 4 GPUs \cite{zhang2013asynchronous}.

A pipeline training approach was propoased in \cite{chen2012pipelined} and a 3.3x speedup was achieved using 4 GPUs, but this
method does not scale beyond number of layers in the neural network.

A speedup of 6x to 14x was achieved using 16 GPUs on training convolutional neural networks \cite{coates2013deep}. In this approach,
each GPU is responsible for a partition of the neural network. This approach is more useful for image classification where 
local structure of the neural network could be exploited.

Distributed model averaging using CPUs is proposed in \cite{zhang2014improving}, and a further improvement 
is done using natural gradient \cite{povey2014parallel}. Our approach is mainly based on this idea, and the main 
difference between our work and this one is that we are performing neural networks training via MPI. Also,
we introduce butterfly-mixing strategy for model averaging.

Butterfly mixing was proposed in \cite{zhao2013butterfly} to interleave communication with computation. In each
reduce period, information (gradient or model) is exchanged between each node and its friend node. The pair of nodes
are shuffled after every reduce so that information in each node can propagate to all other nodes in 
$log(numNodes)$ periods.

In Section 2 we introduce the model approach. In Section 3 we introduce different reduce strategies. Section 4 
explains NG-SGD. Section 5 records experimental results on different setups and Section 6 concludes.

\section{Data parallelization and Model Averging}
Stochastic gradient descent is a popular method for DNN training, even though this is a non-convex optimization 
problem. Since the size of training data is large, minibatch based SGD is often used for model training. 
Roughly, the larger the minibatch size, the higher the converge rate. However, a large minibatch requires
more compute and memory usage, so a straight forward idea for parallelization would be distributing the 
gradient computation to different computing nodes. In each step, the gradient is reduced to one computing node,
averaged and then used to update models in each node. This method can compute the gradient accurately, but 
it requires heavy communication between nodes.

However, if we choose to reduce the weight rather than gradient, it is not necessary to communicate that often. There
is straight forward theory that guarantees convergence, but we do observe reasonable results in experiments.


\section{Model Averaging Strategy}
\subsection{All-reduce}
All-reduce strategy collects the weights from all the nodes in the network. The communication is bounded by 
the bandwidth. Fig~\ref{fig:allreduce} is an example of all-reduce with 4 nodes. One node is responsible for
collecting all the model parameters. It will average the weights and then broadcast it to all the nodes. 
This approach compute the average weight based on all the information in computing nodes. Thus, we expect 
the converge speed to be the fastest.
\begin{figure}[htb]
  \centering
  \includegraphics[width=0.42\textwidth]{allreduce.jpg}
  \caption{All-reduce network}
  \label{fig:allreduce}
\end{figure}

\subsection{Butterfly mixing}
Butterfly mixing is a reduction strategy proposed in \cite{zhao2013butterfly}. It reduces the communication load in 
every iteration: one node would only send and receive message from one other node. The pairing of nodes are shuffled after
each reduction so that information collected in each nodes will spread to all other nodes. Fig~\ref{fig:butterfly} 
is an example of butterfly mixing with 4 nodes. Butterfly mixing requires much less communication than all-reduce 
strategy. However, each node in the network only collects information from two computing nodes at a time.
It takes $log(numNodes)$ communication to spread the message to the whole network. Thus, the converge speed 
of butterfly mixing would be slower than all-reduce strategy. 
\begin{figure}[htb]
  \centering
  \includegraphics[width=0.42\textwidth]{butterfly.jpg}
  \caption{Butterfly mixing network}
  \label{fig:butterfly}
\end{figure}

\subsection{Data exchange analysis}

\section{Natural Gradient for Model Update}
In stochastic gradient descent (SGD), the learning rate is often assumed to be a scalar $\alpha_t$ that may change over time,
the update formula for model parameters $\theta_{t}$ is
\begin{equation}
\theta_{t+1} = \theta_{t} + \alpha_t g_t
\end{equation}
where $g_t$ is the gradient.

However, according to Natural Gradient idea \cite{murata1999statistical,roux2008topmoumoute}, it is possible to replace 
the scalar with a symmetric positive definite matrix $E_t$, which is the inverse of the Fisher information matrix.
\begin{equation}
\theta_{t+1} = \theta_{t} + \alpha_t E_t g_t
\end{equation}

Suppose $x$ is the variable we are modeling, and $f(x;\theta)$ is the probability or likelihood of $x$ given parameters $\theta$, then the
Fisher information matrix $I(\theta)$ is defined as
\begin{equation}
\frac{\partial}{\partial\theta}\log f(x;\theta)
\end{equation}

For large scale speech recognition, it is impossible to estimate Fisher information matrix and perform inversion, 
so it is necessary to approximate the inverse Fisher information matrix directly. We do not include too much 
detail here because this is not the main focus of this report. Details about implementation of NG-SGD could 
be found in \cite{povey2014parallel}.


\section{Experimental Results}
\subsection{Setup}

\subsection{Switchboard}

scaling factor
\begin{figure}[htb]
  \centering
  \includegraphics[width=0.5\textwidth]{scaling.png}
  \caption{Scaling factor and speedup factor v.s. number of gpus}
  \label{fig:scaling}
\end{figure}

percentage of time spent on communication
\begin{figure}[htb]
  \centering
  \includegraphics[width=0.5\textwidth]{percent.png}
  \caption{Percentage of time on MPI communication v.s. number of gpus}
  \label{fig:percent}
\end{figure}


\section{Conclusion and Future Work}
Neural network training can be efficiently speed up by using model averaging techniques and NG-SGD. 
Butterfly-mixing and ring reduction can reduce communication time a lot, but the models may not converge as well as 
Allreduce when the number of jobs goes up to 16.


CUDA aware MPI, learning rate schedule.


\section{Acknowledgements}
We would like to thank Karel Vesely and Daniel Povey who wrote the original "nnet1" neural network training code
and natural gradient stochastic gradient descent upon which the work here is based. We would also like to thank Nelson 
Morgan, Forrest Iandola and Yuansi Chen for their helpful suggestions.

% References should be produced using the bibtex program from suitable
% BiBTeX files (here: strings, refs, manuals). The IEEEbib.bst bibliography
% style file from IEEE produces unsorted bibliography list.
% -------------------------------------------------------------------------
\bibliographystyle{IEEEbib}
\bibliography{paper}

\end{document}
